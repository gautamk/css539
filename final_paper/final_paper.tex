\documentclass[12pt,conference]{IEEEtran}
%\documentclass[letterpaper,twocolumn,10pt]{article}

\usepackage{graphicx}
\usepackage{url}
\usepackage[usenames]{color}
\usepackage{listings}
\usepackage{dirtytalk}
\usepackage[boxed]{algorithm2e}


%------------------------------------------------------------------------- 
% take the % away on next line to produce the final camera-ready version 
%\pagestyle{empty}

%------------------------------------------------------------------------- 
\begin{document}

\title{Limited use cryptographic tokens in securing ephemeral cloud servers}


%for single author (just remove % characters)
\author{
{\rm Gautam Kumar, Prof Brent Lagesse}\\
Computing and Software Systems\\
University of Washington Bothell\\
\{gautamk,lagesse\}@uw.edu
} % end author

\maketitle
\thispagestyle{empty}


\begin{abstract}


Many enterprises and consumers today are dependent on services which deploy their computational resources on Infrastructure as a Service (IaaS) cloud providers such as Amazon's AWS, Google's Cloud platform and Windows Azure. Cloud deployments at scale can have hundreds of virtual servers running and the same time sharing sensitive configuration information such as DB passwords and API Keys. This paper aims to propose a risk limiting architecture to safeguard against a potential information leak of sensitive configuration data using a Central Trusted Authority with hash chains as an authentication mechanism for ephemeral servers in the cloud to provide moving target defence against attackers.
\end{abstract}

\section*{Introduction}

The essence of securing cloud systems is using multiple layers \cite{panwar_layered_2011} of security to increase an attacker's cost for taking over the system. One of the possible layer of security is using moving target defences \cite{evans_effectiveness_2011}. 

In this paper we propose an implementation of moving target defence using ephemeral servers and a central trusted authority which acts on behalf an ephemeral server and proxies requests to sensitive resources such as database servers, caching servers and REST end points. Hash chains are used as an authentication mechanism by the central trusted authority. We take advantage of the limited use property of hash chains to secure authenticate ephemeral servers for a limited period of time.

\section*{Background}
% \subsection*{Cryptographic hash function \cite{rogaway_cryptographic_2004}} 
% A cryptographic hash function is any one way function which meets the following requirements 
% \begin{itemize} 
% \item Preimage resistance
% \item Collision resistance
% \item Second Preimage resistance
% \end{itemize}
% 
% A hash function has preimage resistance if given a hash value $h$ it is computationally infeasible to find any message $m$ such that $h = hash(k,m)$ where $k$ is the hash key.
% 
% A hash function is collision resistant if, given two messages $m_{1}$ and $m_{2}$ it is hard to find a hash $h$ such that $h = hash(k,m_{1}) = hash(k,m_{2})$ where $k$ is the hash key.
% 
% A hash function has second pre-image resistance if given a message $m_{1}$ it is computationally infeasible to find a different message  $m_{2}$ such that $hash(k,m_{1}) = hash(k,m_{2})$ where $k$ is the hash key. The second pre-image resistance is a much harder property to achieve for hash functions. This property is closely related to the birthday problem \cite{lesser_exploring_1999}.
% 
% \begin{figure}[hbtp]
% \includegraphics[scale=0.58]{hash_function.png}
% \caption{A simplified view of a hash function which represents its input and potential result. The length of the hash sum always remains the same regardless of the input size. Any small change in the input drastically changes the output.}
% \end{figure}
% 

\subsection*{Hash Chains \cite{horne_hash_2011}}

Leslie Lamport \cite{lamport_password_1981} first proposed the use of hash chains in his paper on a method for secure password authentication over an insecure medium.

\say{A hash chain is a sequence of values derived via consecutive applications of a cryptographic hash function to an initial input. Due to the properties of the hash function, it is relatively easy to calculate successive values in the chain but given a particular value,it is infeasible to determine the previous value} 

A hash chain in essence is merely the successive computation of a Cryptographic hash function on a given value. 

As an example, Let $x$ be the initial password a hash chain of length 2 would be $H^{2}(x) = H(H(x))$. A hash chain of $n$ values is denoted as $H^{n}(x)$ and the $i^{th}$ value in the chain would be computed as $x_{i} = H(x_{i-1})$.

For a given value in the chain $x_{i}$ its computationally infeasible to determine the previous value in the chain $x_{i-1}$.

\section*{Motivation}

Many organisations are moving their server infrastructure away from physical hardware on-site to virtual hardware in the cloud. Such a migration brings a wide range of problems and potential solutions. 
TODO: ?? 

\section*{Potential threats}

According to OWASP's Top 10 security threats, \say{Sensitive data exposure} is the $6^{th}$ most critical type of security threat in web applications as of 2013 \cite{wichers_owasp_2014}. 

Sensitive data exposure simply refers to unintended exposure of sensitive information such as passwords, social security numbers, date of birth and so on. In the context of a cloud systems sensitive information may also include credentials to access a database, email server, REST API keys and so on. These credentials are usually stored as part of a configuration file which cloud servers can use to authenticate themselves with third party services within or outside the private cloud network. 

According to a report by Risk Based Security \cite{risk_based_executives_2014} \cite{shu_privacy-preserving_2015} the number of data leaks has dramatically increased from 2012 to 2013, to the tune of \$812 million. Though sensitive data exposure in the context of cloud configurations would only constitute a small part of these leaks, leaking of such credentials can potentially lead to massive data leaks or other potential vulnerabilities being exposed to potential attackers.

Sensitive data exposure can potentially be a consequence of other threats such as cross site scripting (XSS) \cite{louw_blueprint:_2009}, Injection is the most critical threat while XSS is the $3^{rd}$ most critical threat as classified by OWASP in 2013 \cite{wichers_owasp_2014}. 

\section*{Proposed solution architecture}

The proposed architecture to defend against the threat of sensitive data exposure is to use a Central Trusted Authority (CTA) responsible for storing sensitive information. The CTA would act as a proxy and would make requests on behalf of client facing servers, refer Fig. \ref{fig:architectureoverview}.

\begin{figure}[hbtp]
\includegraphics[scale=0.3]{overview_architecture.png}
\caption{Architectural overview of the system. This figure describes the three primary modules involved. The client facing server, The central trusted authority and a sensitive resource.}
\label{fig:architectureoverview} 
\end{figure}

The client facing servers would use hash chains to authenticate with the CTA. Hash chains cryptographically limit the number of times a key can be used. Limited use was intentionally selected to promote the creation of a moving target for attackers.

\subsection*{Assumptions}

\begin{figure*}[!ht]
  \centering
  \includegraphics[keepaspectratio=true,scale=0.8]{cta_architecture}
  \caption{Architecture of the Central Trusted Authority. The CTA consists of a storage backend, a hash chain verifier and a request proxy.}
  \label{fig:ctaarchitecture}
\end{figure*}


Client facing servers are the servers which are exposed outside the private cloud network environment. These client facing servers could potentially be load balancing servers, compute servers.

The client facing servers are assumed to be ephemeral. This is common in many cloud deployments \cite{vaquero_dynamically_2011} and contributes to the moving target nature of the security architecture. Companies such as Netflix expect this behaviour with their chaos engineering architecture \cite{basiri_chaos_2016}. This allows for higher reliability of their server infrastructure.

Client facing servers are expected to shut down after their hash chain expires. This contributes to the ephemeral nature and also to moving target defence of the overall system.

\subsection*{CTA Architecture and Implementation}

The Central Trusted Authority consists of three primary components, A hash chain verifier, A storage backend and a request proxy. The CTA generally performs three roles within the system which are

\begin{itemize}
\item Create new hash chain
\item Verify hash chains
\item Proxy requests
\end{itemize}

\subsubsection*{Creating new hash chain}
Hash chains are created by iteratively hashing a secret key $K$, $n$ number of times. 
After the hash chain is created the CTA stores $H^{100}(K)$ in the storage backend and returns $K$ and $n$ to the client facing server. The secret key $K$ is not stored by the CTA.

The client facing server can now use the the secret key $n$ number of times.

\begin{algorithm}
\SetAlgoLined
\caption{Generating a Hash Chain}
\label{algo_generating_hash_chain}
\LinesNumbered
\KwData{Hash Chain secret $K$ and Hash chain length $N$}
\KwResult{Hash Chain $H^{N}(K)$}
$i \leftarrow 1 $\;
$H^{1}(K) \leftarrow H(K)$ \;
\While{$i <= N$}{
	$i\leftarrow i + 1$\;
	$H^{i}(K)\leftarrow H(H^{i-1}(K))$\;
}
\KwRet{$H^{i}(K)$ where $i$ equals $N$} \;
\end{algorithm}
\subsubsection*{Hash chain verification}
Hash chains are used to authenticate client facing server requests which require access to a sensitive resource. The hash chain verification is detailed in algorithm \ref{algo_hash_chain_verification}.

\begin{algorithm}
\label{algo_hash_chain_verification}
\SetAlgoLined
\caption{Verification of Hash Chain authentication}
\LinesNumbered
\KwData{Authentication key $H^{i-1}(K)$ from the client}
\KwResult{Response from sensitive resource}
	Let $H_{client}^{i} = H(H^{i-1}(K))$ \;
	Let $AuthenticationData = fetch(H_{client}^{i})$ \;
	
	\eIf{AuthenticationData exists in storage backend}{
		replace $H_{client}^{i}$ with $H^{i-1}(K)$ in storage backend \;
		fetch and return response from sensitive resource \;
	}{
		return authentication failure \;
	}
\end{algorithm}


\section*{Request Proxy}

The proposed solution architecture of using a Central Trusted Authority (CTA) to proxy requests on behalf of all the clients, places the CTA as a single point of failure. We can overcome this limitation by improving the reliability and trustability of the CTA.

\subsection*{Reliability}

A lot of research \cite{dutta_smartscale:_2012} \cite{vishwanath_characterizing_2010} \cite{xuejie_reliability_2013} \cite{kumar_dynamic_2012} has been conducted into improving the reliability of cloud systems through vertical, horizontal scaling and automatic provisioning. Much of that work can be leveraged for the purpose of improving the reliability of the proposed architecture. For example, Beltr{\'a}n \cite{beltran_automatic_2015} proposes an architecture of utilizing multi-tired load balancers to improve the reliability of a service. 

Large scale database deployments typically deploy databases behind a reverse proxy for load balancing and Geo distribution \cite{hoff_7_2012}. Thus a reverse proxy such as Vitess \cite{hoff_7_2012} is ideally suited to act as the CTA in such architectures.

\subsection*{Security and Trust}

Trust and security of the CTA is another primary area of concern with a distributed cloud system. The CTA as a central point of access to all sensitive resources would be a prime target for malicious actors and thus cannot be completely trusted. Chen et al \cite{chen_towards_2012} propose a solution to this problem of a Malicious proxy using trusted hardware such as Trusted Platform Module (TPM) or the IBM 4758 cryptographic coprocessor \cite{parno_bootstrapping_2010}. 

Chen et al assume the proxy, the CTA in our architecture, is malicious but is incapable of modifying the underlying hardware. The CTA executable is also verified by a trusted third party to operate correctly as a proxy as described in \cite{parno_bootstrapping_2010}.

There are three primary attacks that a malicious actor may perform on the CTA. First, the attacker may try to expose the sensitive information stored, such as API keys and DB passwords, from the CTA using vulnerabilities in the CTA executable. Second, the attacker could potentially modify the requests / responses which are being proxied by the CTA. Third the attacker could potentially launch a reboot attack to inject a malicious executable after attestation to carry out one of the above attacks.

Prior work \cite{libert_tracing_2008, mccune_flicker:_2008} on preventing reboot-attacks can be leveraged to impede the attacker's ability to inject a malicious executable. The proposed architecture also allows for the CTA to be placed behind secure corporate firewalls to further limit the risk of a malicious take over by an attacker. Further work is needed to fully secure the CTA against a malicious proxy.


\begin{figure*}[!ht]
  \centering
  \includegraphics[keepaspectratio=true,scale=0.8]{sequence_diagram}
  \caption{Sequence diagram describing the authentication and proxying capabilities of the CTA. CFS refers to Client Facing Server while CTA refers to Central Trusted Authority}
  \label{fig:ctaarchitecture}
\end{figure*}


\section*{Performance testing}

\begin{figure}[h]
\includegraphics[scale=0.365]{performance}
\caption{Linear time complexity of the hash chain initialization. Execution platform Intel i5-5200U 2.2Ghz, 12GB RAM, SHA-512 implemented in python.}
\label{fig:performance}
\end{figure}

In our testing hash chain initialization showed a linear increase in time complexity as shown in figure \ref{fig:performance}. This is in line with our expectations for hash chain implementations. A hash chain of length 100,000 takes 150 - 200 ms to initialize based on the storage backend used. 

\section*{Limitations}
The proposed architecture and implementation are ideally suited to cloud IaaS providers such as Amazon AWS where ephemeral servers are easily managed. Deploying the proposed system to a cloud provider without the ability to quickly provide large number of ephemeral servers  could result is significant performance degradation. 

Hash chains inherently posses certain vulnerabilities such as a Hash chain cycle and Hash chain length oracle attacks which can potentially reduce the effectiveness of the proposed architecture \cite{lee_hash_2007}.

\section*{Future work}

Formulating an ideal balance between server lifespan and hash chain length to optimize computational resources in a cloud system, this can be derived from current work into load balancing in cloud systems \cite{randles_comparative_2010}. Such a formulation can be used by dev ops engineers in choosing an ideal hash chain size based on the performance requirements of a cloud system.

Integrating Timed Release Cryptography \cite{chalkias_timed_2006} as a hash chain renewal mechanism to potentially increase the lifespan of a server after a certain cool off period.

A Merkle hash tree implementation middle-ware can potentially provide hierarchical authentication authorization \cite{yi_cloud_2012} capabilities to the CTA.

\section*{Related work}

Confidant \cite{lyft_confidant:_2015} is a library maintained by Lyft, a transportation network company based out of San Francisco. Confidant provides an implementation of the Central trusted authority server with encryption at rest, authentication and authorization handled by AWS's Key Management Service, KMS and a storage backend of DynamoDB. This severely hampers the ability of a potential user to deploy a Confidant instance on a different cloud IaaS provider beside Amazon's AWS. 

Confidant also serves as an inspiration for the CTA component of the proposed architecture.

\section*{Summary}

In this paper we propose an architecture to provide moving target defence and minimize the damage that attackers can cause by centralizing the storage of sensitive configuration information and taking advantage of the limited use property of hash chains to authenticate potentially vulnerable client facing servers. 

\bibliographystyle{plain}
\bibliography{css539_class_see}

\end{document}


\documentclass[10pt,conference]{IEEEtran}
%\documentclass[letterpaper,twocolumn,10pt]{article}

\usepackage{graphicx}
\usepackage{url}
\usepackage[usenames]{color}
\usepackage{listings}
\usepackage[caption=false]{subfig}


%------------------------------------------------------------------------- 
% take the % away on next line to produce the final camera-ready version 
\pagestyle{empty}

%------------------------------------------------------------------------- 
\begin{document}

\title{Enhancing the security of cloud server configuration with hash chains}


%for single author (just remove % characters)
\author{
{\rm Gautam Kumar}\\
Computing and Software Systems\\
University of Washington Bothell\\
gautamk@uw.edu
} % end author

\maketitle
\thispagestyle{empty}


% At most, 1 page that briefly summarizes your project proposal.  After reading 
% this, a reviewer should understand why you want to do this work, what your 
% general approach is, and what the benefits will be when you complete the work.
\section{Executive Summary}
Information security is a difficult problem, especially in cloud computing environments where 
threats to information security are difficulty to identify and mitigate. One such problem is 
managing the sensitive configuration information.

In traditional enterprise deployments developers often stored configuration within code or on config 
files along with code. This was considered a safe practice because the hardware and Operating System 
where code deployment occurred was owned and managed by the enterprise themselves. These systems 
were usually behind a strong firewall so threats were much easier to mitigate. 

In cloud computing environments, where hardware and the underlying software hypervisor are shared 
amoung thousands of customers who could potentially be using the resources of a single data center, 
Threats to security are much more complex and we need a layered strategy to secure our systems. In 
this environment storing sensitive configuration information on disk could potentially be dangerous.


% Detail side channel attacks

One of the ways developers have secured systems in this environment is to use a centralised trusted 
server to store and retrieve sensitive configuration information. The goal of this project is 
investigating ways to improve the security and allow for forward secracy using Hash 
Chains\cite{tian_self-healing_2008}.

\section{Project Description}

% Activities
% Background
% Survey
% Broader impact
% Why technically rigorous
The project description must include the activities that you plan for your 
research.  Provide sufficient background so that a reviewer can understand 
what you are going to do.  Provide a sufficient survey of related work so that 
a reviewer understands what work has been done before and where your work will 
fit in the state of the art (for example, if reputation systems have been 
built for well-connected networks, but they rely on assumptions that no longer 
hold in a new type of network, tell the reviewer about the existing state of 
the art and identify why your new reputation system must be created because 
the old ones won't work in the new network). You also must include a section 
on Broader Impacts that describes why your topic is important to the world and 
a section on Intellectual Merit that describes why the problem you are 
addressing is technically rigorous.


\bibliographystyle{plain}
\bibliography{CSS539}

\end{document}

